\chapter{Introduction}\label{sec:1}
\epigraph{\itshape Linguists have not formulated a ``methodology of sentence judgments.''\\[-2\baselineskip]}{\citep{vanRiemsdijk1986}}
\section{Goals}\label{sec:1.1} 
I aim to demonstrate in this book that grammaticality judgments and other sorts of linguistic intuition, while indispensable forms of data for linguistic theory, require new ways of being collected and used. A great deal is known about the instability and unreliability of judgments, but rather than propose that they be abandoned, I endeavor to explain the source of their shiftiness and how it can be minimized. I argue that if several simple steps are taken to remove obvious sources of bias, grammaticality judgments can provide an excellent source of information about people's grammars. Thus, I respond to two of the most widespread criticisms of generative grammar\schdash{}namely, that it involves constructing theories of intuition rather than of language use, and that it is highly subjective and biased by the views of the linguist. This involves drawing from a wide range of literature and from linguistic theory (both pro- and antigenerative) and from the philosophy of language. Linguists can expect to take away from this book numerous practical
suggestions on how to collect better and more useful data, and on how to respond to criticisms of such data. As I set out to review almost all the major psycholinguistic experiments that have been done to investigate the linguistic judgment process, psycholinguists should also find much of interest, including numerous suggestions for experimental work that they might wish to pursue.

Throughout much of the history of linguistics,  linguistic  intuitions  have been the most important source of evidence in  constructing  grammars.  Major types of intuition include canonical grammaticality judgments, intuitions about derivational morphological relationships among words, intuitions about correspondences among different utterance types (e.g., question/answer  pairs), identifications%
% 2  \textsuperscript{Chapter} \textsuperscript{One}
 of structural versus lexical ambiguity, and discriminations of the syntactic status of superficially similar word strings, among many others (\citet{Chomsky1985}). While I most often talk about grammaticality judgments  in this book, I treat this as a cover term,\is{grammaticality judgment@\textit{grammaticality judgment}!as cover term} because these judgments have received much more attention than other kinds of linguistic intuition. It should be understood that wherever possible I intend the discussion to extend to other sorts of intuition, and I do not wish to imply that grammaticality judgments in the narrow sense have any special status.

It is not immediately obvious why a description of people's competence\is{competence (grammatical)} in understanding and producing language should be based on behavior in situations where they are doing neither but, rather, are reporting intuitions. There are four key reasons for the use of grammaticality judgments. First, by eliciting judgments, we can examine reactions to sentence types that might occur only very rarely in spontaneous speech or recorded corpora. This is a standard reason for performing experiments in social science\schdash{}observational study\is{observational study, difficulties of} does not always provide a high enough concentration of the phenomena we are most interested in.\footnote{In principle, the conclusion does not automatically follow. One could in theory do experiments on the production and comprehension of sentences chosen by the researcher, without recourse to judgments. In practice, however, this is problematic. On the production side, it is difficult to induce a subject to produce precisely the sentence one wishes to study without actually exposing the subject to the sentence. On the comprehension side, it is hard to discover anything about the nature, or even the success or failure, of the comprehension process without eliciting some additional reaction, such as a judgment.}
%\textsuperscript{1}
 A second, related reason for using grammaticality judgments is to obtain a form of information that  scarcely exists within normal language use at all\schdash{}namely, \isi{negative information}, in the form of strings that are not part of the language. The third reason for using judgments is that when one is merely observing speech it is difficult to distinguish reliably slips,\is{production errors/slips} unfinished utterances, and so forth, from grammatical production. A fourth and more controversial reason is to minimize the extent to which the communicative and representational functions of language skill  obscure our insight into its mental nature. Thus, we construct arbitrary situations for adults to deal with, which tap the structural properties of language without having any real function \citep{Bever1986}. This last rationale presupposes a particular view of grammatical competence as cognitively separate from other facets of language knowledge and use, and hence its validity depends on one's theoretical stand on this issue. The first three reasons, however, are relatively theory-neutral. (See \citet{Grandy1981} for these and other standard arguments in favor of the use of grammaticality judgments; see \citet[62\textendash{}63]{Newmeyer1983}, for additional arguments against the use of alternative data sources.)

Such justifications seem sensible enough, perhaps even unavoidable, but that has not stopped some skeptics and critics from wanting to abandon the use of judgments altogether: ``I ... regard the `linguistic intuition of the native speaker' as extremely valuable heuristically, but too shifty and variable (both from speaker to speaker and from moment to moment) to be of any criterial value'' \citep[15]{Householder1965}. \citet{Gethin1990} believes that grammaticality judgments are useless. Becker finds their very lack of communicative function problematic:

\begin{quote}
And so the ``modern'' linguist spends his or her time starring or unstarring terse unlikely sentences like ``John, Bill and Tom killed each other'' (to pick one at random from a recent journal), which seethe with repressed frustration and are difficult to work into a conversation. These example sentences bear no discernible resemblance to the sentences which compose the text that purportedly explains them\schdash{}yet the linguist's own sentences are also alleged (implicitly) to be drawn from the same English Language! \citep[70]{Becker1975}
\end{quote}

In response to such attitudes, some philosophers of language have adopted positions that have gone farther in the opposite direction than most theoreticians would likely feel comfortable with. For example, \citet{Carr1990} states:

\begin{quote}
The arguments that intuitively accessed grammaticality judgments either are not sufficient or are not necessary as the evidential basis for linguistic theory cannot proceed, and the fact of theoretical linguistic practice shows that autonomous linguistics proceeds with such evidence being not only necessary but also sufficient for the testing of hypotheses. (p. 57)

  
%%please move the includegraphics inside the {figure} environment
%%\includegraphics[width=\textwidth]{lspschuetzeges-img1.png}
 \end{quote}

\noindent
As I make clear in this book, I do not believe that one can defend the sufficiency of judgments  alone.

Regardless of what the critics say, it is clear that the use of grammaticality judgments is here to stay for the foreseeable future. Still, soliciting linguistic judgments is problematic in a number of respects. Not only is the elicitation situation artificial, raising the standard issues of ecological validity, but the subject is being asked for a sort of behavior that, at least on the face of it, is entirely different from
%\originalpage{4} %  Chapter One
everyday conversation.\footnote{\citet{Householder1971,Householder1973} tried to find the closest conversational analogues to grammaticality judgments. He suggests that the following are some of the typical reactions one is inclined to have when a speaker utters something out of the ordinary: the listener is baffled (``I don't get it; come again''); the listener finds an inconsistency or implausibility (``You must mean X, don't you?''); the listener characterizes the speaker as being from another dialect area (``Aha, a southerner!''); the listener concludes that the speaker is quoting a proverb or poetry (``You mean `figuratively speaking,' I suppose'').
}
%\textsuperscript{2}
 This has led some to suggest that theoretical linguists are in fact constructing grammars of linguistic intuitions or judgments, which need not be identical with grammars of the competence\is{competence (grammatical)} underlying production or comprehension \citep{Bever1970a,Birdsong1989,GleitmanEtAl1979}. However, Wayne Cowart (personal communication)\ia{Cowart, Wayne} argues that linguistic judgments do play a fairly central role in our day-to-day lives, and cites the following examples. We might use judgments of other people's speech when we first meet them in forming opinions about them and categorizing them on various dimensions. We might assess other people's utterances with respect to our own grammar (and vice versa) in order to manipulate the extent to which we are perceived as belonging to the same community. Children growing up in a multilingual environment might judge the utterances they hear in order to assess which language they most closely resemble, allowing them to differentiate languages they are learning concurrently. The last of these three suggestions strikes me as the most compelling, but given how little we understand about multilingual acquisition, we cannot say with certainty that children's evaluations of utterances are similar to the explicit judgments of adults.

In addition to these problems, which are often found in psychology as well, there are important shortcomings that arise because linguistic elicitation does \textit{not} follow the procedures of psychological experimentation. Unlike natural scientists, linguists are not trained in methods for getting reliable data and determining which of two conflicting data reports is more reliable. In the vast majority of cases in linguistics, there is not the slightest attempt to impose any of the standard experimental control techniques, such as random sampling\is{experimental design!subjects, random sampling of} of subjects and stimulus materials\is{experimental design!stimulus materials, random sampling of} or counterbalancing for order effects.\is{experimental design!order, counterbalancing of} (See \citet{Derwing1979} for a discussion of linguists' ``blatantly informal'' methods.) Perhaps worst of all, often the only subject in these pseudoexperiments is none other than the theorist himself or herself: ``One of the unfortunate consequences of Chomsky's\ia{Chomsky, Noam} mentalist view of linguistics is that in recent years a number of younger linguists have indulged very heavily in arguments based on their intuitions about quirks of their personal
idiolects''\footnote{Such behavior is certainly not a consequence of the Chomskian\ia{Chomsky, Noam} view in the sense that he encourages or implicitly endorses it. If there is any causal link at all between the theory and such practices, it presumably arises from the mistaken belief that if the object of study (grammar) is in the mind of the individual, then the behavior of a single individual (e.g., oneself) constitutes the only data one need consult. I discuss throughout the book why this does not follow.}\is{dialects (idiolects)}
%\textsuperscript{3}
 \citep[74]{Sampson1975} (see also \citealt{Newmeyer1983} and \citealt{BradacEtAl1980}). In the absence of anything approaching a rigorous methodology, we must seriously question whether the data gathered in this way are at all meaningful or useful to the linguistic enterprise. More than a few observers of linguistics have agreed with Labov's\ia{Labov, William} ``painfully obvious conclusion ... that linguists cannot continue to produce theory and data at the same time'' \citep[199]{Labov1972a}. What is to stop linguists  from (knowingly or unknowingly) manipulating the introspection process to substantiate their own theories?\footnote{One possible answer is that competition among linguists will prevent such manipulation; see \chapref{sec:4}, fn. 19.}
%\textsuperscript{4}


The informal nature of judgment  collection  has long been  acknowledged. Consider, for example, the following passages from \citet{Chomsky1969}:

\begin{quote}
The gathering of data is informal; there has been very little use of experimental approaches (outside of phonetics) or of complex techniques of data collection and data analysis of a sort that can easily be devised, and that are widely used in the behavioral sciences. The arguments in favor of this informal procedure seem to me quite compelling; basically, they turn on the realization that for the theoretical problems that seem most critical today, it is not at all difficult to obtain a mass of crucial data without use of such techniques. Consequently, linguistic work, at what I believe to be its best, lacks many of the features of the behavioral sciences. (p. 56)


I have no doubt that it would be possible to devise operational and experimental procedures that could replace the reliance on introspection with little loss, but it seems to me that in the present state of the field, this would simply be a waste of time and energy. (p. 81)
\end{quote}

\noindent
Derwing's response to this attitude is unequivocal.

\begin{quote}
Such `arguments' are not compelling at all. The choice here is between proven data-collection methods and the reliable   `hard' data to which they lead or inferior `informal' methods and the `soft' data which inevitably result. ... This is hardly a choice. In linguistics there is reason to believe that the choice is \textit{available}, but has been ignored or neglected in the rush to theory. ... All that is necessary is `to replace intuition by some more rigorous criterion' ([\citealt[24]{Chomsky1962}]) and attempt to establish, under controlled experimental conditions, whether naive native speakers really can do all the things which Chomsky\ia{Chomsky, Noam} says that they can (such as make consistent judgments of grammaticality). \citep[250]{Derwing1973}
\end{quote}

\noindent
The conflict between these two positions is precisely what this book is about.

An additional rationalization for the use of grammaticality judgment data in some cases seems to have been related to Chomsky's\ia{Chomsky, Noam} competence/per\-form\-ance distinction (see \sectref{sec:2.2} for a detailed discussion of this matter).\is{competence (grammatical)!versus performance} Actual speech production and comprehension are supposedly fraught with errors of all kinds, such as false starts, and are subject to human memory limitations. These so-called performance variables serve to obscure a speaker's underlying competence.\is{competence (grammatical)} But what if we could relieve subjects of the ``cognitive burden'' of actual production or comprehension and present them with ready-made sentences such that the only task would be to judge their grammaticality? Would this not allow us to get much closer to people's true competence?\footnote{\citet{Sampson1975} phrases the position as follows, although he goes on to reject it: ``The part of our brain which makes conscious judgments about the English language perhaps has a `hot line' to the part of our brain which controls our actual speaking, so that we know what we can and cannot say in English in the same direct, `incorrigible' way that, say, I know I have toothache'' (p. 72). I have not found many explicit examples of this reasoning in the theoretical linguistic literature, but the belief seems to have been very widely held, because there are numerous instances (cited in \citet{Birdsong1989}) where Lasnik,\ia{Lasnik, Howard} Chomsky,\ia{Chomsky, Noam} and others attempt to curb this view. For example, \citet[20]{Lasnik1981} states that ``grammaticality judgments are often \textit{incorrectly} considered as direct reflections of competence'' (emphasis added). Certainly, many authors have wrongly accused Chomsky\ia{Chomsky, Noam} of claiming that people have a consistent ability to assess grammaticality (e.g., \citealt{Nagata1988}). \citet[11]{GleitmanEtAl1970} attribute to Chomsky\ia{Chomsky, Noam} the claim that \textit{linguists'} judgments are free of contamination. The view might have stemmed in part from confusion of Chomsky's\ia{Chomsky, Noam} terms \textit{intuition} and \textit{judgment}, a matter that I take up in \sectref{sec:2.2}.}
%\textsuperscript{5}
 Unfortunately, there is ample evidence that it would not. While grammaticality judgments offer a \textit{different} access path from language use to competence, they are themselves just another sort of performance \citep{Birdsong1989,LeveltEtAl1977,Bever1970b,Bever1974,BeverEtAl1971,Grandy1981}, and as such are subject to at least as many confounding factors as production, and likely even more.


 
The purpose of this chapter is to motivate the search for resolutions to the issues raised above and to outline the approach to be taken. The discussion will be mostly at an informal, conceptual level, with technical terminology and details left for subsequent chapters. The structure of the chapter is as follows. In \sectref{sec:1.2}, I use the problems raised above, along with others, to motivate the goals and approach of the remainder of the book. Before intuitions (or any other behavior) can really begin to tell us something about competence, we need at least to be aware of, and ideally to understand the effects of, the component psychological processes that intervene between the two. I propose that this understanding is achievable in principle if we construct a comprehensive model of the judgment process. This model would allow the extensive research already conducted by psycholinguists to be unified and integrated, and would allow contradictory results to be scrutinized. At the very least, a well-supported model of this type should raise the awareness of linguists to the vast complexities underlying the apparently simple task of deciding whether a sentence is grammatical. In \sectref{sec:1.3}, I further motivate the endeavor by describing some real examples of linguistic research that show how the approach I propose can work to benefit the field. \sectref{sec:1.4} presents a working hypothesis concerning the source of extragrammatical influences on judgments that I assume in much of what follows. Finally, \sectref{sec:1.5} sets out the scope and structure of the remainder of the book.

\section{Approach} \label{sec:1.2} 
\begin{quote}\itshape Linguistic intuitions became the royal way into an understanding of the competence which underlies all linguistic performance.  However, if such a linguistic competence exists at all, i.e., some relatively autonomous mental capacity for  language, linguistic intuitions seem to be the least obvious data on which to base the study of its structure. They are derived and rather artificial psycholinguistic  phenomena  which develop late in language acquisition ... and are very dependent on explicit teaching and instruction. They cannot be compared with primary  language use such as speaking and listening. The  domain of Chomskian\ia{Chomsky, Noam} linguistics is linguistic intuitions. The relation between these intuitions and man's capacity for  language, however, is highly obscure.\\[-2\baselineskip]\begin{flushright}\upshape\citep{LeveltEtAl1977}\end{flushright}
\end{quote}
%\todo{need to keep citation on same page as epigraph}

In this section I describe briefly the motivations for and approach to an in-depth investigation of the process of forming grammaticality judgments, which will be expanded upon in later chapters. I argue that an understanding of this process would provide the basis for an objective method of establishing which judgment data bear most directly on the grammar, and of extracting grammatical information from judgments that are confounded by other factors. The idea of factoring grammaticality out of acceptability judgments has been proposed before (e.g., \citet{Birdsong1989,CarrollEtAl1981,Botha1973}). In the words of \citet{GleitmanEtAl1970}, ``if we could strip away various contaminating factors in behavior, we might see the grammar bare'' (p. 10). That contaminants are present and in need of stripping will be demonstrated below. The traditional view of how judgments relate to language use is too naive. For instance, Cohen, while he shows considerable concern for the issues, in the end remains overly simplistic:

\begin{quote}
A native speaker's intuition that the string \emph{S} is grammatical is just his immediate and untutored (though in principle observable) inclination to take \emph{S} as being well-formed, and in this sense he can have such an intuition if and only if he would be (equally observably) inclined to utter \emph{S} whenever his circumstances, motivation, beliefs, etc., are precisely appropriate for a communication with the sense of \emph{S} and also he is applying ideal standards of care and attention in the linguistic formulation of his utterance. It follows that the difference between an utterance of \emph{S} and an intuition of \emph{S}'s grammaticalness, as data for grammar, is just that while the former constitutes an actual occurrence of \emph{S} in human speech, the latter establishes a potential occurrence\schdash{}i.e., a potential production by some speaker. Hence intuitions of grammaticalness can always provide a vital kind of data that actual utterances may often fail to present; and because of this it is exclusive reliance on the observation of actual utterances, not reliance on intuitions of grammaticalness, that fails to mirror essential features of scientific method. \citep[240\textendash{}241]{Cohen1981}
\end{quote}

If one is concerned with the scientific method, it seems sensible to begin the way other scientists do, by scrutinizing the data source. \citet{Bever1972} makes an appropriate analogy to natural science in this regard:

\begin{quote}
Such investigations are analogous to that of a biologist who checks the limits on a microscope before examining single cells with it (for example, if he does not know the refractory limitations of his microscope he may spuriously attribute color bands to the cells). However, to explore the limits of the available tools of observation is not to suggest that cells do not exist. Similarly, I have tried to examine the limits on the most extensive observational tool linguists utilize to gather
data about linguistic structure: grammaticality intuitions. This investigation does not suggest that linguistic structure does not exist; indeed the investigation of interactions between manifest intuitions and inner linguistic structure cannot proceed without the \textit{a priori} assumption that the inner structure is itself as ``real'' as the expressed intuitions. (p. 412)
\end{quote}


It should  not be controversial to suggest that linguists ought to study their methodology for these standard scientific reasons: to get more reliable facts by developing methods for gathering, processing, and reporting data so that the results of different investigators are comparable and their methods of analysis consistent; and to get more valid data by assessing what errors are present in the data reports and trying to eliminate their sources \citep{Labov1978}. In \chapref{sec:2}, I present several examples showing that these measures are now necessary. The days are over when linguistics had more than enough to worry about with uncontroversial, commonplace judgment data, and the sophisticated and complex judgments now in use by theoreticians assume much about human abilities that remains unproved, even unscrutinized. We simply do not know whether the questions we are asking people are meaningful and can be answered in any principled way. I argue below that there is much to be gained by applying the experimental methodology of social science to the gathering of grammaticality judgments, and that in the absence of such practices our data might well be suspect. Eliminating or controlling for confounding factors requires us to have some idea of what those factors might be, and such an understanding can only be gained by systematic study of the judgment process. Finally, I argue that by studying inter-speaker variation rather than ignoring it (by treating only the majority dialect or one's own idiolect),\is{dialects (idiolects)} one uncovers interesting facts.

This general approach is not a new proposal; Levelt\ia{Levelt, {Willem J.M.}} et al. and Bever\ia{Bever, {Thomas G.}} have articulated the general direction of this approach with great foresight:

\begin{quote}
Where do grammaticality intuitions come from? It makes no sense to assume a priori that the domain of linguistic intuition is a relatively closed one, as many linguists appear to do. Such intuitions are highly dependent on our knowledge of the world and on the structure of our inferential capacities. \citep[89]{LeveltEtAl1977}

\textit{What is the Science of Linguistics a Science of?} Linguistic intuitions do not necessarily directly reflect the structure of a language, yet such intuitions are the basic data the linguist uses to verify his grammar. This fact could raise serious doubts as to whether linguistic science is about anything at all, since the nature of the source of its data is so obscure. However, this obscurity is characteristic of every exploration of human behavior. Rather than rejecting linguistic study, we should pursue the course typical of most psychological sciences; give up the belief in an ``absolute'' intuition about sentences and study the laws of the intuitional process itself. (\citealt[346]{Bever1970a}; emphasis in original)
\end{quote}


\citet{ElliotEtAl1969} make the case for studying variation: ``There are facts both about linguistic theory and about the grammars of particular languages whose existence will be obscured unless variation is taken into account'' (p. 52); ``At least some variation is not completely mysterious and seems amenable to statement in terms within the realm of linguistic theory. At the same time, linguists have a responsibility to determine what kinds of variation exist rather than ignoring variation by basing syntactic descriptions on trivially small numbers of informants''\is{experimental design!subjects, number of} (p. 58). Carden (e.g., \citeyear{Carden1973}) makes the same case. These authors go on to show that variability on theoretically important issues such as the \textit{do so} construction and reflexive anaphors\is{Binding Theory} falls into implicational hierarchies of acceptability.

Thus, the approach that I pursue in this work is to examine the process of judging grammaticality, including the role of grammar in this process and its relation to other relevant mental components. In addition to studying an intriguing form of behavior, one that has been almost entirely overlooked in favor of production and comprehension, I attempt to integrate the existing research findings in this area by sorting out the facts from the specific theories proposed in each study; assessing their consistency; clarifying how they fit into an overall theory of cognition; establishing which methodologies are most reliable, valid, and informative; and proposing new experiments to fill gaps in our knowledge. While the psychology of grammaticality judgments might hold as many complexities and and mysteries as language itself, that is no reason for despair or dismissal\schdash{}it is all the more reason for us to begin the task of unraveling them.

\section{Motivation: Whither Linguistics?}\label{sec:1.3} 

A glance at the length of the reference section of this book shows that more than a few language researchers have concerned themselves with the problems that I am addressing here. Many of the experimental findings were published a number of years ago, but experimental research seems to be on the increase again, along with continued calls for greater use of formal experimentation for collecting judgment data (e.g., \citet[100\textendash{}101]{Hirst1981}). Does all this work have any real effect on the way theoretical linguistics is carried out on a day-to-day basis? While instances in which theoretical linguistics takes experimental research into account are still few and far between, I believe that issues in grammaticality judgment collection and interpretation \textit{are} receiving greater attention. From among the studies that make appropriate use of judgment data within the framework of theoretical argumentation I will cite three examples of what I consider to be cutting-edge work in the hope of facilitating and encouraging more research along these lines.

The first such work is by \citet{GrimshawEtAl1990} (building on work by \citet{ChienEtAl1990}), who argue that, contrary to first appearances, children's linguistic behavior does tell us something about their grammars\schdash{}namely, that they include Principle B of Binding Theory.\is{Binding Theory!acquisition of} 
Their reasoning is that

\begin{quote}
performance in an experiment, including performance on the standard linguistic task of making grammaticality judgments, cannot be equated with grammatical knowledge. To determine properties of the underlying knowledge system requires inferential reasoning, sometimes of a highly abstract sort. (p. 188)

The inevitable screening effects of processing demands and other performance factors do not prevent us from establishing the character of linguistic knowledge; they just make it more challenging. ... An analysis \textit{of} these performance factors makes it possible to see, if only
dimly, through the performance filter. (p. 217)
\end{quote}

Grimshaw \& Rosen conclude that, while children do not show perfect mastery of Binding Theory, they perform above chance, and treat violations of \isi{Binding Theory}\is{Binding Theory!acquisition of} differently from nonviolations. They argue that inherent properties of the relevant constructions, as well as of the experiments by which they are evaluated, conspire to worsen children's performance,\is{children, eliciting judgments from} especially as reflected in their apparent lesser mastery of Principle B versus Principle A. The paper is unusual in that it represents work by theoreticians in which a major goal is the explanation of the connection between behavior on judgment tasks and linguistic knowledge. While a naive view of the facts contradicts their claim, they argue that once psychological factors such as \isi{response bias} and experimental demand characteristics\is{demand characteristics (of experiments)} are taken into account, the results support their theory. One may still dispute their conclusions, but their effort points in the right direction.


% Chapter One

The second example of work that uses judgment data appropriately is a paper by \citet{CardenEtAl1981}, the goal of which is to establish structural conditions on pronoun coreference. Carden \& Dieterich deal with cases where a pronoun precedes the noun phrase with which it is coreferent, e.g., examples \REF{ex:1:1} and \REF{ex:1:2}, where cosubscripting indicates coreference:


\ea\label{ex:1:1}
I knew him$_i$ when Harvey$_i$ was a little boy.
\z


\ea\label{ex:1:2}
 We'll just have to fire him$_i$, whether McIntosh$_i$ likes it or not.
\z

\noindent
A handful of instances of these constructions have been found in texts, but proportionately very few compared to cases of uncontroversial \isi{backwards coreference} \is{Binding Theory!and backwards coreference} like that in \REF{ex:1:3}:

\ea\label{ex:1:3}
The boy who loves her$_i$ claims that Mary$_i$ is a genius.
\z

\noindent
\citet{Langacker1969}, who claims that sentences like \REF{ex:1:1} and \REF{ex:1:2} are bad, pairs such a sentence with a clearly good example in his paper, whereas \citet{Reinhart1976}, who claims that sentences like \REF{ex:1:1} and \REF{ex:1:2} should be good, contrasts\is{contrast effect} such a sentence with a clearly bad one. This issue, according to Carden\ia{Carden, Guy} \& Dieterich,\ia{Dieterich, Thomas} also illustrates the problem with \isi{corpus data}: ``How do we interpret this data? Do we cheer because there \textit{were} six examples, and conclude that Reinhart was right? Or do we boo because there were \textit{only} six examples, as against hundreds of the uncontroversially good type? ... We may have a good but (accidentally) rare construction; or we may have a bad construction occurring a few times because of errors'' \citep[591]{CardenEtAl1981}. The authors investigate the status of sentences like \REF{ex:1:1} and \REF{ex:1:2} using an experiment that shows that these questionable forms are accepted no more often than an uncontroversially bad form. (In each case, only 1 of their 30 subjects accepted them.) The materials were constructed so that a preceding context sentence allowed a plausible reading where the crucial coreference relationship did \textit{not} hold, as well as a reading where it \textit{did} hold, so that subjects would not be forced by considerations of plausibility  into accepting an ungrammatical structure. They also tested the uncontroversially  bad sentences preceded by the same context sentence, so that the results would be fully comparable. The one significant shortcoming of their methodology is that they employed only two examples of each type of crucial sentence, so their results  might have been affected by quirks of those specific sentences.

A third exemplary study also involves \isi{backwards coreference}.\is{Binding Theory!acquisition of} It was conducted by \citet{GerkenEtAl1986}, who were apparently not aware of Carden\ia{Carden, Guy}
and Dieterich's\ia{Dieterich, Thomas} work in this area. On the basis of inter-speaker differences in the interpretation of the same sorts of sentences, Gerken\ia{Gerken, LouAnn} \& Bever\ia{Bever, {Thomas G.}} propose that linguistic universals, and \isi{Binding Theory} in particular, are not necessarily applied to complete sentence structures as given by linguistic competence but, rather, are applied to the speaker's \textit{perceived} structure as generated during sentence processing. They point out that for many sentences it is not necessary to compute a complete syntactic structure in order to extract the meaning, and suggest that this computation might therefore be delayed until after the initial parse, or might never be carried out at all. Gerken\ia{Gerken, LouAnn} \& Bever\ia{Bever, {Thomas G.}} are specifically concerned with \isi{Binding Theory}'s prediction that there should be a strong contrast between VP-attached and S-attached subordinate clauses with regard to potential backwards coreference, such that \REF{ex:1:4}, in which the complement clause is under the VP, should be much worse than \REF{ex:1:5}, in which the adverbial clause is attached to the S node, at least under certain versions of the theory.


\ea\label{ex:1:4}
The dog told him$_i$ that the horse$_i$ would fall.
\z


\ea\label{ex:1:5}
The dog hit \textit{him$_i$} while the horse$_i$ ate lunch.
\z

However, Gerken\ia{Gerken, LouAnn} \& Bever's\ia{Bever, {Thomas G.}} acceptability experiment failed to find any such overall difference.\is{backwards coreference} They argue that there are no general surface cues for the difference between S-node versus VP-node attachment, so it is possible that the distinction is not made in on-line parsing structures. In fact, there is a tendency for English speakers to segment sentences after a noun-verb-noun sequence, and those subjects who performed strong perceptual closure at this juncture (as revealed by another experiment) did not make the attachment distinction for pronouns, whereas those who made less use of the closure strategy did make the predicted contrast between \REF{ex:1:4} and \REF{ex:1:5}. Subjects who exhibited strong closure did not have a VP node accessible for attachment when they got to the subordinate clause, because the VP had been closed off, and they therefore treated all such clauses as S-attached, allowing coreference in both sentence types. Thus, these individual differences do \textit{not} require us to posit individual differences in the formulation of \isi{Binding Theory}. Besides the possibility that complete trees are never computed, an alternative interpretation suggested by Gerken\ia{Gerken, LouAnn} \& Bever\ia{Bever, {Thomas G.}} is that we do compute full constituent structures but cannot access them for certain tasks, being left instead with the perceptual structure alone. This raises the intriguing but rather unlikely possibility that linguists have developed introspective techniques to get at these fuller structures, while untrained speakers have not. The lesson to be
drawn
%\originalpage{14} %  Chapter One
from these three studies is that theoretical linguistics can benefit from a
concern for the  judgment process.

\section{A Working Hypothesis} \label{sec:1.4}

In this section I set out my own basic working hypothesis regarding the interaction of metalinguistic\footnote{See \chapref{sec:3} for attempts at a definition of the term \textit{metalinguistic}.} performance factors and the grammar\is{grammar!versus performance} in determining grammaticality judgments.\is{competence (grammatical)!versus performance} My hypothesis is a reaction to countless studies that have demonstrated that grammaticality judgments are susceptible to order and context effects, handedness differences, etc., and have then concluded, on the basis of this manipulability (or on the basis of the gradience of judgments, or on other properties), that the grammar itself must have these properties, or that these properties must be part of the language-specific component of the brain. Such conclusions are not justified. In my view, we should start from the position that the entire behavior of making grammaticality judgments is the result of interactions between primary language faculties of the mind and general cognitive properties, and crucially does \textit{not} involve special components dedicated to linguistic intuition. Thus, my hypothesis is that for any effect on a language (judgment) task, there could be an analogous effect on a similar nonlinguistic\is{nonlinguistic behavior, parallels with language} cognitive (judgment) task. I have parenthesized the word \textit{judgment} to indicate that I suspect that the truth of this hypothesis extends beyond judgments to other metalinguistic tasks, although they will not be my concern here. In other words, my claim is that \textit{none} of the variables that confound metalinguistic data are peculiar to judgments about language. Rather, they can be shown to operate in some other domain in a similar way. (This is quite similar to \citegen{Valian1982} claim that the data of more traditional psychological experiments have all the same problems that judgment data have.) It is not always easy to find convincing instances of such effects in other domains, however. The most likely candidates would be judgments in another sensory modality, such as taste, smell, or vision,\is{visual judgments} which at least at a low level are unlikely to involve the language facilities of the mind. I will suggest just two arbitrary examples of cognitive domains that might be affected by the same variables that affect linguistic tasks.

First, in the visual domain,\is{visual judgments} shape recognition and judgments of size, numerosity, etc., are potential candidates for parallels with linguistic tasks.\is{nonlinguistic behavior, parallels with language} \citet{BergumEtAl1979boje,BergumEtAl1979jebo} have found that in judging visually ambiguous figures (e.g., \isi{Necker cubes},\ia{Necker, {Louis Albert}} \isi{Rubin vase figures},\ia{Rubin, Edgar}  and \isi{Jastrow rabbit-duck figures})\ia{Jastrow, Joseph} certain individuals experience reversals much more frequently than others. One might predict that these people also detect linguistic ambiguity more easily than others.\footnote{The two types of individuals were architecture majors and business majors, respectively. The authors do not draw a conclusion as to whether the difference in reversal perception might be due
to
an innate tendency toward perceptual instability, or might be a learned ability.}
%\textsuperscript{7}
 Second, in the perfume industry, experts\is{expert versus naive judgments} are employed to smell products that are to be marketed and to test for certain properties that nonexperts in this field have never heard of.\is{olfactory judgments} These experts might differ from naive perfume smellers in the same ways that linguists differ from naive sentence judges.\is{naive versus expert judgments}\is{expert versus naive judgments} Wherever possible in the following chapters, I draw parallels of this sort between experimental results in psycholinguistics and known effects in other fields, or I propose a search for such effects. Such findings could greatly assist us in factoring out these effects from our grammatical judgment data, bringing us closer to an accurate picture of linguistic knowledge.

My hypothesis represents common-sense expectations about the relation between language and other behaviors, and empirical support for it would thus not be particularly surprising \citep{Bever1970a}. However, even if the hypothesis is supported, it still does not explain \textit{how} cognitive principles and linguistic knowledge come to interact in the mind to produce linguistic judgments. There are (at least) two possible extreme interpretations. It could be that properties such as context dependence and susceptibility to training effects belong to separate modules of the mind that are implicated in judgment behavior but not in other forms of behavior (e.g., a decision-reporting component). At another extreme, it could be that these properties are inherent in the cognitive substrate on which language and all other higher cognitive functions are built. Both possibilities have important implications that go far beyond the present work. My intuition is that each is probably true of some properties, but it will not be possible to settle the issue here. In principle the two explanations are empirically distinguishable, since the modular theory predicts that there could be behaviors that circumvent the modules in question and do not show the relevant effects, whereas the substrate theory predicts that they are everywhere and inescapable. (These arguments
are of course drastically oversimplified.) If we should find that for a given effect
there seems to be no parallel elsewhere in human cognition,\is{nonlinguistic behavior, parallels with language} then and only then would we have the beginning of an argument for the special nature of linguistic judgment among human knowledge systems.

%\originalpage{16} %  Chapter One

\section{Scope and Organization}\label{sec:1.5}

I do not attempt in this book to treat the subject of grammaticality judgments in its entirety. Rather, I restrict my investigation on two somewhat arbitrary, but fairly sensible, dimensions. First, in asking what grammaticality judgments are judgments \textit{of}, I look only at the acceptability (and grammaticality) of word strings; i.e., I consider only syntactic, as opposed to phonological, wellformedness, although in a broad sense acceptability/grammaticality often entails conformity to the phonology as well as to the syntax, and even to other linguistic components. Second, while several sorts of experiment are potentially relevant to the subject of grammaticality judgments, I systematically exclude a number of subject populations. There will be little mention of the judgments of second language learners and other nonnative speakers, except when they bear on our understanding of native intuitions. Only a passing glance will be cast on the \textit{development} of metalinguistic awareness (as it bears on adult awareness), which has become virtually a field unto itself. And no data from aphasic speakers or others with language impairments are considered. Putting it positively, I focus mostly on the syntactic grammaticality judgments of ``typical'' adult native speakers. Furthermore, I try to emphasize work on intuitions about general structures rather than work that bears only on the use of particular lexical items or constructions, since the former are generally taken to be more fundamental indicators of core linguistic knowledge.

Other sources cover parts of this territory, and the reader may wish to consult them. \citet{Newmeyer1983} devotes a chapter of his book to the data base of linguistic theory, but his goal is to defend, rather than to (constructively) criticize, the generative modus operandi, and I disagree with many of his conclusions, although I cite many of the same sources. \citet{Chaudron1983} deals only with experimental psycholinguistic  work, but provides  a useful  summary  chart of many of the studies I discuss,\footnote{To compare the results of previous studies on the basis of Chaudron's\ia{Chaudron, Craig} chart would be misleading, however; the experiments differed in ways too subtle and too complex for his categorizations to capture.} and examines many procedural details that I omit;\footnote{It will become apparent that my reports of experimental work are often concerned with two particular features of elicitation experiments, the instructions that are given to subjects, and the evaluation scheme (rating scale, categories, ranking procedure, etc.) that is used. The importance of these two factors is discussed in detail in Sections \ref{sec:5.2.1} and \ref{sec:3.3.4},
respectively. Variation in these two features is perhaps the biggest reason why virtually no two studies of grammaticality judgments are directly comparable.} however,
at least half of his paper is devoted to studies of second-language learners. \citet{Labov1975} takes a position quite sympathetic with my own, but is concerned mostly with sociolinguistic variation. While much of the experimental work he discusses is not directly relevant to the issues discussed in this book, his methodological proposals have heavily influenced my own. Finally, \citegen{Birdsong1989} review of the literature, which occupies two of his chapters, overlaps considerably with mine, but lacks the sort of principled overall organization that I attempt to provide. His aim, like Chaudron's,\ia{Chaudron, Craig} is to apply discoveries about grammaticality judgments to issues in second-language learning and teaching research. Nonetheless, many of his methodological proposals have been incorporated here. Thus, none of the major previous studies of grammaticality judgments have attempted, within the basic framework of generative grammar, to explain why grammaticality judgments behave the way they do and to propose changes in the way that linguists treat judgment data. That is what I attempt to do in this book.

The book is organized as follows. In \chapref{sec:2}, I summarize the history of the concepts of grammaticality and acceptability and their associated notations, focusing on the ways in which grammaticality judgments are used by syntactic theorists today and arguing that such uses demand a careful examination of judgments, not as pure sources of data but as instances of metalinguistic performance. I also consider where their use fits in the broader scheme of introspection and intuition in social science. \chapref{sec:3} is a discussion of several important issues that arise when a performance view of grammaticality judgments is taken: tasks one can use to elicit them, scales one can use to report them, how people might go about giving them, and how and what they might tell us about linguistic competence. Chapters 4 and 5 cover the major body of psycholinguistic research that has been devoted to discovering ways in which the judgment process can vary systematically with differences between subjects (\chapref{sec:4}) and experimental manipulations (\chapref{sec:5}). \chapref{sec:4} considers individual differences in two major categories: endogenous, or organismic; and exogenous, or experiential. \chapref{sec:5} examines task factors in two major categories: stimulus materials, or what is to be judged; and procedural methods, or how it is to be judged. In reviewing the literature in these two chapters, I attempt wherever possible to point out parallels with other cognitive behaviors. \chapref{sec:6} represents the integration of the substantive and methodological findings and discussions of Chapters \ref{sec:3}\textendash{}\ref{sec:5}. I present a preliminary model of the judgment process that reflects what is known about linguistic intuitions, and I propose methods for collecting grammaticality judgments that avoid the pitfalls of previous work and take into account the factors that have
% 18  \textsuperscript{Chapter} \textsuperscript{One}
been shown to influence judgments. Readers who seek immediate practical advice on the collection of judgment data may wish to consult \sectref{sec:6.3} directly; it does not assume familiarity with preceding material. \chapref{sec:7} considers ways in which mainstream linguistic theory might be affected by the growing body of research on grammaticality judgments and suggests directions that could be pursued to advantage in future studies.