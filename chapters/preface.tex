\addchap{Preface (2016)}
\begin{refsection}
Since the original version of this book (University of Chicago Press, 1996) went out of print in the 2000s, I have continued to receive inquiries from people asking how they can obtain a copy. I am therefore thrilled that Language Science Press has offered to make the title available again, as part of their Classics in Linguistics series. I would like to thank series editors Stefan Müller and Martin Haspelmath, as well as Sebastian Nordhoff and Felix Kopecky, for their help in making this happen.

The content of this new printing is identical to the first printing, with the following exceptions:
\begin{itemize}
\item I have altered the wording in a few places where I found it insufficiently clear or terminologically outdated;
\item my uses of the term \textit{informant(s)} have been replaced with \textit{consultant(s)} or \textit{speaker(s)}, in keeping with current practice (of course, the former term still appears in some quoted passages);
\item I have updated the reference information for a couple of works that had not been published at the time of the original printing, particularly \citet{Cowart1997};
\item the original index has been split into name and subject indexes, and both are now more comprehensive.
\end{itemize}

\noindent In terms of presentation, the following things have changed: 
\begin{itemize}
\item the format of citations and references has been adapted to LSP house style, as have other minor typographical choices;
\item full given names have been added to references whenever available;
\item since the text has been freshly typeset, the page numbers do not match those of the original printing; however, the (sub)section numbers are unchanged: I suggest using those if it is necessary to specify a location within a chapter. Example numbers are also unchanged.
\end{itemize}

Importantly, I have \textit{not} attempted to update the content in light of subsequent relevant research, since this would undoubtedly have compelled me to try to write a whole new book. Of course, linguistics and psycholinguistics have chang\-ed a great deal in the 20 years since I  completed the original manuscript; e.g., ``theoretical'' linguistics has notably become more ``experimental.'' Also, some of my own views on the issues have evolved over those two decades. There are passages in the book that I would have omitted or altered, \textit{if} I had allowed myself to make any substantive revisions. Instead, I have chosen to restrict all follow-up discussion to this preface. In what follows I try to point readers to works that should allow them to ``get up to speed'' on intervening developments.

For collections that are comprised mainly of papers on topics that are important in the book, see \citet{McNair1996}, \citet{Penke2004}, \citet{Kepser2005}, \citet{Borsley2005}, \citet{Featherston2007} and replies in the same journal issue, \citet{Featherston2007a}, \citet{Featherston2009}, and \citet{Winkler2009}. My more recent views can be found in the following surveys: \citet{Schuetze2006,Schuetze2011} and \citet{Schuetze2013}.

There have been (at least) four major developments involving the empirical base of linguistics that anyone interested in the topic should be aware of. 
    
\begin{enumerate}
\item The adaptation of the \isi{magnitude estimation task} from \isi{psychophysics} to judgment collection \citep{BardEtAl1996}. This was touted as having numerous potential advantages over the traditional \isi{Likert scale}\ia{Likert, Rensis} task, most or all of which have been subsequently refuted (see \citealt{Weskott2011} and \citealt{Sprouse2013}). 
\item The use of World Wide Web searches to establish attestation, and infer acceptability, of certain sentence/construction types. I discuss the limitations of this approach in \citet{Schuetze2009}.
\item The use of Amazon Mechanical Turk (AMT) and potentially other crowdsourcing platforms as sources of subjects for acceptability judgment and many other psycholinguistic experiments (so far, in only a handful of languages). For an empirical investigation of how AMT results compare with judgments collected in the lab (on a small range of constructions in English), see \citet{Sprouse2011}.
\item Detailed empirical challenges to\schdash{}and defenses of\schdash{}the proposal, advocated in \sectref{sec:7.2} in the book, that Subjacency\is{Subjacency violation} effects could be reduced to processing factors. See \citet{Yoshida2014} and the Stanford/Maryland debate (\citet{Hofmeister2010,Hofmeister2012,Hofmeister2012a,Sprouse2012,Sprouse2012a}, and many of the contributions in \citealt{Sprouse2014}).
\end{enumerate}
 
Finally, there is a statement by Chomsky,\ia{Chomsky, Noam} which I attribute in the book (p.~\pageref{ChomskyPrefacestart}) to a popular press source, about which I have often been questioned, wherein Chomsky calls it a truism that genetically based \isi{Universal Grammar} (UG) is subject to \textit{some} individual variation.\is{interspeaker variation!in Universal Grammar} For those who have asked whether Chomsky’s position can be confirmed in any academic publications, I offer the following quotes:

\begin{quote}
Putting aside genetic variation (an interesting but marginal phenomenon in the case of language) and conceivable but unknown epigenetic effects, the principles of UG, whatever they are, are invariant. \citep[35]{Chomsky2013}
\end{quote}

\begin{quote}
It is hardly controversial that [the faculty of language] is a common human possession apart from pathology, to an approximation so close that we can ignore variation. \citep[138]{Chomsky2008}
\end{quote}

\noindent I am aware of no empirical evidence that would indicate how much UG can vary across individuals.\\\\
\begin{minipage}{.45\linewidth}
	\begin{flushleft}
		\noindent Carson T. Schütze 
	\end{flushleft}
\end{minipage}
\begin{minipage}{.54\linewidth}
	\begin{flushright}
		\noindent December 2015
	\end{flushright}
\end{minipage}
\nocite{Phillips2006}
\phantomsection\addcontentsline{toc}{section}{References}
\printbibliography[heading=prefacebib]
\end{refsection}