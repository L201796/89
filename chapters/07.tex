\chapter{Looking Back and Looking Ahead}\label{sec:7}

 \epigraph{\itshape Recent trends in linguistic research have placed increasing dependence on relatively subtle intuitions. ... Subtle intuitions are not to be trusted until we understand the nature of their interaction with factors that are irrelevant to grammaticality. If we depend too much on such intuitions without exploring their nature, linguistic research will perpetuate the defects of introspective mentalism as well as its virtues.\\[-2\baselineskip]}{\citep{Bever1970b}}

 \section{Introduction}\label{sec:7.1}

 The epigraph from Bever above concurs very well with my own findings in the preceding six chapters. By way of a response, it seems fair to say that the field has begun taking steps to explore the nature of grammaticality judgments, and I hope that the present work makes its own contribution to that exploration. In this final chapter I concentrate mostly on what lies ahead in this endeavor.

 I will not attempt to summarize the discussion to this point in any detail, but will very briefly review the structure of the argumentation and illustrate that a major intention has been to provide substantive support for the views of grammaticality judgments that have been expressed succinctly and eloquently by previous researchers in this area. It might be hoped that their observations will carry more weight with the underpinnings of the extensive experimental and theoretical literature that this book has assembled.

 In Chapters \ref{sec:1} and \ref{sec:2}, I reviewed some of the history of how the concept of grammaticality has evolved since the 1950s, various opinions on its empirical status, and how it is used and misused today among theoretical linguists. On this basis I argued that theory is no longer being based on clear cases,\is{clear cases!insufficiency of} and that detailed study of the judgment process is therefore required to establish how to deal with unclear cases. \citet{Botha1973} extends the argument even further: ``consider the status of the so-called clear cases of linguistic intuitions. Today, it can be seen that
 % 203
 %\originalpage{204} %  Chapter Seven
 native speakers may make, concerning a particular linguistic property of some sentence, judgments which are at once, clear, decisive, and consistent without there necessarily being genuine linguistic intuitions at the basis of these judgments'' (p.~205). I also pointed out that concern with these problems has been sorely lacking to date in linguistic work. As Pullum somewhat cynically puts it, ``The median number of speakers on whom the entire corpus of examples in an English syntax paper is checked before publication, including its author, is zero'' \citep[453]{Pullum1987}. I made the specific proposal that the sources of the perturbations in grammaticality judgments exist independent of the language faculties of the mind. In retrospect, that position has probably turned out to be too strong. Some of them might be attributable specifically to the parser, for instance. Nonetheless, the more phenomena we can reduce to language-independent sources, the better, by Occam's razor, so I maintain that one should always seek evidence for this position first. \chapref{sec:3} was devoted to pursuing the suggestion that grammaticality judgments be studied as an instance of (meta)linguistic performance: as part of a larger family of such tasks, as an instance of graded behavior, perhaps as an instance of introspective behavior, and as just one more source of evidence about grammars. By this point, it was already apparent that ``in many ways, intuition is less regular and more difficult to interpret than speech'' \citep[199]{Labov1972a}.

 Chapters \ref{sec:4} and \ref{sec:5} were devoted to detailed examinations of the range of causes for variability in judgments. \chapref{sec:4} was concerned with the degree of variation between subjects and its attribution either to inherent characteristics or to life experiences. \chapref{sec:5} examined task factors, which were broken down by being (mostly extragrammatical) features of the sentences being judged, or features of the procedure used to elicit judgments. It is clear that in neither case do we have a full understanding of the way these factors work, so that Birdsong's\ia{Birdsong, David} plea still stands: ``thorough study of the psychological and epistemological intricacies of metalinguistic performance is necessary if we are to achieve an understanding of the linguistic knowledge it is often thought to reflect'' \citep[49]{Birdsong1989}. Finally, \chapref{sec:6} was an attempt to integrate these findings in terms of an abstract view of the implied mental structures and a proposed methodology for more rigorous data collection among linguists. The componential view of the judgment process as involving many more pieces than language use lent credence to another of Birdsong's\ia{Birdsong, David} statements: ``the hypocrisy of rejecting linguistic performance data as too noisy to study, while embracing metalinguistic performance data as proper input to theory, should be apparent to any thoughtful linguist'' (p. 72). If it was not apparent before, it should be now!

 % Looking Back and Looking Ahead  205

 The format of the remainder of this chapter is straightforward. In \sectref{sec:7.2}, I consider the sorts of research that naturally follow most directly from the present work, including both experimental and theoretical undertakings. I conclude in \sectref{sec:7.3} with some speculation about the future in the field of linguistics, specifically about the role that grammaticality judgments are likely to play down the road, and the chances that attitudes toward their 
 collection and application will change. 

 \section{Directions for Further Research}\label{sec:7.2}

 As acknowledged in \sectref{sec:1.5}, what I have presented here is far from a complete picture of the state of the art in studying grammaticality judgments. There is great potential for elucidating many of the issues we have confronted by considering kinds of data that have been excluded here. For example, experiments involving people with amnesia could clarify the memory mechanisms underlying structural priming effects, repetition effects, context, etc. Research on the development of metalinguistic skills in children should tell us more about the interdependence of these skills and primary language skills.\is{children, eliciting judgments from} Work with second-language learners should help to establish the relationship between intuitions and use as skill in the language increases, while avoiding some of the methodological problems involved in eliciting judgments from children; experiments with aphasics could serve a similar purpose. Finally, more about the process of linguistic judgment in general could be learned by more detailed work on the nature of lexical, phonological, semantic, and pragmatic judgments, in comparison with syntactic ones. The larger open question of the existence of linguistic competence and its role in language processing remains a major unresolved issue in the psychological investigation of language processing.

 As for specific lines of investigation that would follow more directly from the present work, many potentially informative experiments have been proposed in response to specific problems with published research; these will not be repeated here. One major area into which I have not delved deeply is the substantiation of the hypothesis proposed in \sectref{sec:1.4}, namely, that we can find a non-linguistic analog for all of the perturbations that grammaticality judgments are subject to. In many cases we have seen implicitly that this is true to a certain degree. For instance, individual differences correlated with field dependence, handedness, age, sex, creativity, and world knowledge are certainly not unique to linguistic intuitions. Neither are differences due to expertise, which parallel those of
 %\originalpage{206} %  Chapter Seven
 linguistic training and literacy. Likewise, most of the procedural factors considered in \sectref{sec:5.2} are familiar from other branches of psychology, although the particular changes that some of them induce in grammaticality judgments (e.g., repetition and mental state effects) are not obviously analogous to those in other domains. Parallels to other cognitive spheres are hardest to draw in the area of stimulus factors, since many of these are closely tied to the nature of language itself. While at least some types of context effect have perceptual analogs, it is hard to think of a nonlinguistic equivalent of structural well-formedness being affected by meaning, lexical content, morphology, etc. It seems that the best we can do for the moment is to point to domains where supposedly orthogonal features of a stimulus affect judgments of the target feature. Thus, the first major hurdle in this line of work is to find suitable domains of cognition in which to look for such parallels to the manipulability of grammaticality judgments. By way of an example, in addition to possibilities discussed in \chapref{sec:1}, one intriguing area I have come across involves judgments in legal cases. \citet{Kaplan1977}
 reports a number of phenomena that look promisingly parallel to the stimulus effects we have seen. For instance, it has been shown experimentally that jurors are influenced by factors totally irrelevant to the legal merits of a case in ways that depend on the nature of the crime.\is{legal judgments (by juries)}\is{grammaticality judgments!parallels to jury behavior} An attractive defendant will be judged less likely to be guilty of a burglary but more likely to be guilty in a confidence swindle. Personal traits such as race, sex, and marital status have been shown to affect outcomes of cases even when jurors are explicitly instructed not to pay attention to them. Even when jurors are told that a certain variable is or is not statistically a predictor of guilt, they do not use this information in deciding how to treat the data in question.

 A second  major area that cries out for follow-up is the methodology of judgment elicitation itself. The logical next step in the research program would be to design and run case study experiments incorporating the proposals made in \sectref{sec:6.3}, developing a specific set of instructions along the way. Such a study will undoubtedly point out problems with the proposals, suggest refinements, etc., and will allow the resulting data to be assessed for reliability and to be compared with results from more casual data collection. To the extent that the data are more reliable, one of the goals of the exercise will have been met. The third and perhaps most ethereal line of research to follow from the present work would involve finding independent motivation  (outside linguistic judgments) for the components of the model proposed in \sectref{sec:6.2}, for instance the general control strategies. It is not at all obvious how to proceed here.


 % Looking Back and Looking Ahead  207

 I would like to close this section by mentioning some excellent new experimental work that begins to address the three research areas just outlined, with very encouraging results. First I summarize the findings of several independent lines of investigation into the nature of Subjacency\is{Subjacency violation} effects, all of which converge on the conclusion that these should not be attributed to the grammar, but rather to some extragrammatical component, perhaps the parser. Then I discuss work that shows the viability and benefits of taking judgment-gathering methodology seriously. \citet{NevilleEtAl1991} present some event-related brain potential (ERP)\is{ERPs (Event-Related Potentials)} results (see \sectref{sec:3.2}) that could be a step toward using this technology to confirm independently certain kinds of judgment data.\is{brain, as source of grammaticality data} They found, in addition to the frequently observed N400 response to semantic anomaly, distinct responses to certain types of syntactic ill-formedness. Neville et al. tested the following sort of paradigm, wherein the first four sentences are grammatical controls for the remaining ill-formed sentences:\is{picture NP@\textit{picture} NP}

 \ea\label{ex:6:1}
 \ea
 The man admired a sketch of the landscape.
 \ex
 The man admired Don's sketch of the landscape.
 \ex
 What did the man admire a sketch of?
 \ex
 Was a sketch of the landscape admired by the man? 
 \ex
 The man admired Don's of sketch the landscape.
 \ex
 The man admired Don's headache of the landscape.
 \ex
 What did the man admire Don's sketch of?
 \ex
 What was a sketch of admired by the man?
 \z
 \z

 \noindent
 Sentence (\ref{ex:6:1}e) exemplifies a \isi{phrase structure violation} in contrast with (\ref{ex:6:1}b), where
 \textit{sketch} and \textit{of} are in the correct order. Sentence (\ref{ex:6:1}f)
 supposedly
 instantiates semantic anomaly, in contrast with (\ref{ex:6:1}b), although it is not obvious that one could not just as well analyze it as a subcategorization failure, because \textit{headache} cannot take an argument PP and that is the only function the following phrase can serve; hence, it might still embody a syntactic type of ill-formedness. The contrast between  (\ref{ex:6:1}c), which  is fine,  and  (\ref{ex:6:1}g), which  is degraded,  is attributed  to  the specificity of the NP from which the latter extraction has taken place (due to the genitive). Finally, the contrast between (\ref{ex:6:1}d) and (\ref{ex:6:1}h) illustrates the effect of a \isi{Subjacency violation} in the latter. Neville et al.'s basic finding was that only the supposed semantic anomaly generated N400 responses. The \isi{Specificity} violation generated an N125, which was similar to the effect of the phrase structure violation, while the \isi{Subjacency violation} generated a substantially different pattern. The authors take these results
 %\originalpage{208} %  Chapter Seven
 as supporting the notion of a distinct syntactic type of ill-formedness not tied to meaning, and as suggesting that Subjacency\is{Subjacency violation} may be of a different character from the other two types of syntactic violation, perhaps due to processing difficulty rather than grammatical prohibition. These conclusions are obviously speculative, but encouraging nonetheless. The weak link in this study is that it has not yet been shown that different sentence structure types that violate \textit{the same} grammatical constraint yield the same type of ERP\is{ERPs (Event-Related Potentials)} response. For instance, do Subjacency violations\is{Subjacency violation} of the type (\ref{ex:6:1}h) (extraction from a complex NP)\is{Complex NP Constraint} yield the same pattern as those of the type \textit{What did you wonder who bought?} (extraction from a \textit{wh}-island\is{wh-island@\textit{wh}-island})? Until this is found, one cannot exclude the possibility that every sentence type simply yields a distinct pattern of activity. (See \citet{KluenderEtAl1993} for more evidence, from judgments as well as ERPs, bearing on the nature of Subjacency.)\is{Subjacency violation}

 \citet{Snyder1994} has taken a different approach to the same question, conducting experiments into the nature of syntactic \isi{satiation} effects on judgment tasks and what they might tell us about the nature of grammars and the extent of the problem of ``linguists' disease,''\is{linguists' disease (accepting ill-formed sentences)} i.e., the possibility that linguists come to accept ill-formed structures due to repeated exposure. Snyder was able to experimentally induce some satiation effects that were specific to certain types of sentence, i.e., that did not involve an across-the-board change in the subjects' liberality of judgment, but did generalize somewhat across different lexical items within each type. He found that some classes of sentences were susceptible to satiation while others were not. The particular paradigm involved yes/no judgments of grammaticality of 58 sentences, where sentence types recurred over the course of the questionnaire. The measure of satiation was whether a particular subject judged a sentence type grammatical more often later on in the experiment. The types of ungrammaticality\is{islands, syntactic} tested are exemplified in \REF{ex:6:2}.\enlargethispage{1\baselineskip}

 \ea\label{ex:6:2}
 \ea \begin{tabular}{@{}>{\raggedright}p{7.25cm}p{3cm}}Who does John want for Mary to meet? & [\textit{want-for}]\end{tabular}
 \ex \begin{tabular}[b]{@{}>{\raggedright}p{7.25cm}p{3cm}}What does John know that a bottle of fell on the floor? & [\isi{subject island}]\end{tabular}
 \ex \begin{tabular}[b]{@{}>{\raggedright}p{7.25cm}p{3cm}}Who does John wonder whether Mary likes? & [\textit{whether} island\is{whether island@\textit{whether} island}]\end{tabular}
 \ex \begin{tabular}{@{}>{\raggedright}p{7.25cm}p{3cm}}Who does Mary think that likes John? & [\textit{that}-trace]\end{tabular}
 \ex \begin{tabular}[b]{@{}>{\raggedright}p{7.25cm}>{\raggedright}p{3cm}}Who does Mary believe the claim that John likes? & [Complex NP\linebreak Constraint]\end{tabular}
 \ex \begin{tabular}{@{}>{\raggedright}p{7.25cm}p{3cm}}Who did John talk with Mary after seeing? & [\isi{adjunct island}] \end{tabular}
 \ex \begin{tabular}[b]{@{}>{\raggedright}p{7.25cm}>{\raggedright}p{3cm}}How many did John buy books? & [Left Branch\linebreak Constraint]\end{tabular}
 \z
 \z

\is{Complex NP Constraint}\is{Left Branch Constraint (LBC)}\is{Comp-trace effect}\is{want-for violation@\textit{want-for} violation}
 \noindent
Sentence types (\ref{ex:6:2}c) and (\ref{ex:6:2}e) showed statistically significant \isi{satiation} effects (with order of presentation counterbalanced), type (\ref{ex:6:2}b) tended towards such an effect, and no other sentence types showed any effect.

 Somewhat surprisingly, those sentences that did satiate\is{satiation} were not those among the sentence types that seem to show the greatest degree of interspeaker variation in general, namely, \textit{that}-trace\is{Comp-trace effect} (see \sectref{sec:2.3.3}) and \textit{want-for} violations. Snyder suggests that the sentences that \textit{are} subject to satiation may be those whose ill-formedness has its roots in processing rather than in the grammar proper.\is{parsing!versus grammar} Perhaps with repeated exposure, subjects develop alternate parsing strat\-egies\is{parsing!strategies} to deal with Subjacency violations,\is{Subjacency violation!and satiation} and having done so they do not find the sentences as bad. Conversely, violations that do not improve with repeated exposure are therefore less likely to be purely a result of parsing restrictions. This evidence converges with that of Neville\ia{Neville, Helen} et al. from ERP\is{ERPs (Event-Related Potentials)} research as a compelling story in which judgments of badness involving Subjacency may not reflect the grammar at all. It is also interesting that Snyder did find two types of \isi{Subjacency violation} that patterned the same way and differed from the other types of violation he looked at.

 Turning more directly to issues of methodology, \citep[12\textendash{}27]{Cowart1997} conducted a
 series of experiments to test the skeptics' claim that judgments from naive speakers\is{naive subjects (nonlinguists)} would not be useful to linguists even if systematically collected. He focused on the question of whether their judgments were stable,\is{grammaticality judgments!stability of} i.e., whether the relative acceptability of a set of sentences stayed the same across different groups of subjects at different times under different conditions. With a careful experimental design involving counterbalanced orders and multiple instances of each crucial sentence type, several contrasts important to syntactic theory were shown to be highly statistically reliable. In each case, the amount of variability across subjects was much less than that across sentence types, i.e., there was a great deal of agreement on the relevant contrasts. One experiment tested \textit{wh}-extraction from different kinds of \textit{picture} NPs\is{picture NP@\textit{picture} NP} and found that  four sentence types all showed significantly different judgments of grammaticality. In particular,  in contrast to the standard judgment in the literature, an extraction like \textit{Who did the Duchess sell a portrait of?}, while better than \textit{Who did the Duchess sell Max's portrait of?},\is{Specificity} is significantly  worse than a simple adjunct question like \textit{Why did the Duchess sell a portrait of Max?}, a result that was confirmed by comparison with other closely related  sentence types. Cowart\ia{Cowart, Wayne} also demonstrates how these  gradations of acceptability  can be interpreted. A dialect-split interpretation can be excluded
 %\originalpage{210} %  Chapter Seven
 because of the very small amount of variation among subjects for the crucial sentence type, as compared to variation across sentence types. Thus, these sentences do seem to have an intermediate status for individual speakers. Another set of sentences studied by Cowart\ia{Cowart, Wayne} involves \textit{that}-trace effects.\is{Comp-trace effect} Again, he found very stable results, with subject and object extractions ((\ref{ex:6:3}b) and (\ref{ex:6:3}d) below) being equally acceptable without \textit{that}, but subject extractions being much worse in the presence of \textit{that} ((\ref{ex:6:3}a) versus (\ref{ex:6:3}b)), confirming the standard judgments in the literature. However, object extraction is significantly worse with \textit{that} than without it ((\ref{ex:6:3}c) versus (\ref{ex:6:3}d)), a fact that standard theories have not taken into account. (See \citet[12\textendash{}27]{Cowart1997} for some qualifications on this finding.)


 \ea\label{ex:6:3}
 \ea I wonder who you think that likes John.
 \ex I wonder who you think likes John.
 \ex I wonder who you think that John likes. 
 \ex I wonder who you think John likes.
 \z
 \z
 
 \noindent
 It remains to be argued whether the grammar should be responsible for these unexpected contrasts.

 \section{The Future in Linguistics}\label{sec:7.3}

 In Sections \ref{sec:1.3} and \ref{sec:7.2}
 I describe some recent work in the generative paradigm that I feel makes exemplary use of grammaticality judgments. Obviously if this trend of linguists basing their theories on experimental data is to continue and grow, linguists will have to be trained in areas that they traditionally have not been required to know anything about: statistics and experimental design in general, and the psychology of grammaticality judgments in particular. I would echo \citegen{Greenbaum1977c} recommendation that every linguistics department should offer a course in experimental linguistics. In addition to reasons internal to our own field, this would give students a leg up in joining the blossoming interdisciplinary enterprise of cognitive science. It would also seem to be a natural outgrowth of Chomsky's\ia{Chomsky, Noam} own suggestion that linguistics be viewed as a branch of cognitive psychology. Somehow, the focus on cognitive issues has not yet been accompanied by adoption of the scientific standards and concern with methodology of that discipline.\footnote{Noam Chomsky (personal communication)\ia{Chomsky, Noam} believes that research practice in linguistics ought to follow that in the natural sciences, where (in contrast to the social sciences)\is{linguistics!versus social sciences} ``almost no one devotes attention to `methodology'.''\is{methodology in generative grammar} Obviously, I disagree.}
 %\textsuperscript{1}
  But even if only a small proportion of linguists were
 actually to carry out their own experimental data collection, all could benefit by knowing more about problems of experimental bias, individual differences, introspection, etc. The question is whether theoretical linguists are likely to heed such advice. I suggest that part of the reason for linguists' lackadaisical attitude in this regard is not so much that they believe their data are clear-cut, but that there is little motivation for putting effort into a systematic approach because, unlike in most of the social sciences,\is{linguistics!versus social sciences} there is no standard publication format requiring authors to describe how their data were gathered. (\citet{Grandy1981} makes a similar point, and suggests other possible reasons why the deplorable state of lack of rigor continues.) Also, since linguists typically have no training in experimental design, they do not appreciate how useful and important it is. On this question, we can do little more than keep our fingers crossed. It does seem, based on my assessment of the literature, that more and more linguists are coming around.

 \citet{CardenEtAl1981}, who make proposals similar to my own, give a typical response to their work, suggesting that many linguists will oppose such methodological changes. They cite \citet{Green1978} as saying that if proposals like theirs were adopted, ``research would come to a standstill.'' Certainly this would be true if \textit{every} sentence had to be subjected to extensive experimental verification \citep{Labov1975}, but that is unnecessary. If we adopt Labov's\ia{Labov, William} Clear Case Principle (see \sectref{sec:6.3.3}), this will only be required when we have reason to believe that there is disagreement. Green continues with a second objection: ``I doubt if any experimental results, no matter how clean, would affect the status of crucial disputed examples. Linguists will still trust their own intuitions of grammaticality.'' A third objection that I have frequently heard is that much more money will be required to carry out linguistic research under these proposals (Elan Dresher, personal communication).\ia{Dresher, B. Elan} \citet{Ringen1979} believes that is only a rationalization for inaction: ``The cost of data does not explain the traditional reliance on small numbers\is{experimental design!subjects, number of} of informants since many linguistic research projects are extremely well-funded. One suspects that even if time and funds were unlimited, surveying large numbers of informants would be judged an unnecessary and indeed trivial endeavor'' (p. 120, fn. 39). All I can say in response to the latter two objections is, I hope not.

 Somewhat more optimistically, Labov\ia{Labov, William} suggests that introspective linguists are most likely to resort to experimentation on data that are crucial both ways, i.e., that can either clinch their argument or destroy it. This would be a reasonable first step. Certainly, intuitive judgments by native speakers (but, one hopes, fewer and fewer linguists) will not be replaced by other kinds of language behavior as the
 %\originalpage{212} %  Chapter Seven
 major source of data, at least on syntactic questions, in the foreseeable future. But it is perfectly legitimate to keep using judgment data while we attempt to understand them better. While their potential contamination by extraneous factors is an important concern, once we are willing to actively explore the nature of these factors the problem becomes manageable. They might add sufficient noise to obscure actual grammatical phenomena, but they cannot systematically change the pattern of results unless they too are stable. If so, they can be studied directly and then factored out, as people have attempted to do with Subjacency effects\is{Subjacency violation} in the studies mentioned above. The key is that we must always ask ourselves whether a systematic effect we find might be attributable to a combination of grammatical and extragrammatical factors, rather than purely grammatical ones. If the best account we can find of a pattern of variance in judgment data involves some discrete grammatical construct, that is the closest we can come right now to knowing that the mind really embodies such a construct. One finding from the literature should leave us optimistic: relatively few experiments have shown that the \textit{pattern} of results is changed by the various manipulations that have been tried. This should increase our confidence that judgments do tell us about something real and important. It is up to a critic who believes that even carefully collected systematic judgment evidence is distorted and not relevant to grammar to show that that is the case, by being explicit about what the confounding factor is and \textit{how} it distorts the results.\footnote{I thank Wayne Cowart\ia{Cowart, Wayne} for discussion of issues raised in this section.}
 %\textsuperscript{2}


 Linguistics has much to gain and nothing to lose by taking data collection, and particularly judgment collection, much more seriously, both with regard to the insights that will be gained and the theoretical issues that will be clarified, and with regard to the standing of the field as a scientific endeavor in the larger academic setting. The realization seems to be growing that the psychology of grammaticality judgments can no longer be ignored.

 %\nocite{*}

 % bib
 % epigraphs
